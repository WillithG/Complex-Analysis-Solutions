\documentclass[12pt,letterpaper,boxed]{hmcpset}
\usepackage[margin=1in,headheight=14pt]{geometry}
\usepackage{amsfonts, amsmath, amssymb, enumerate, fancyhdr, gensymb, lastpage, mathtools, parskip, graphicx}
\usepackage{xcolor, tikz-cd}
\newcommand{\wg}[1]{\textcolor{violet}{#1}}
\newcommand{\OO}{\mathcal O}
\newcommand{\Q}{\mathbb Q}
\newcommand{\R}{\mathbb R}
\newcommand{\C}{\mathbb C}
\newcommand{\Z}{\mathbb Z}
\newcommand{\abs}[1]{\left|#1\right|}
\newcommand{\im}{\text{im }}
\newcommand{\inv}{^{-1}}
\newcommand{\normal}{\unlhd} %% one can also use \trianglelelefteq
\newcommand{\anglee}[1]{\langle #1 \rangle}
\usepackage[shortlabels]{enumitem}

\newcommand{\pdv}[2]{\frac{\partial #1}{\partial #2}}

% Numbering macros
\pagestyle{fancy}
\lhead{Will Gilroy}
\chead{Algs Homework \#}
\rhead{03 November 2021}
\lfoot{}
\cfoot{}
\rfoot{Page\ \thepage\ of\ \pageref{LastPage}}

\linespread{1.5}

\newcommand\blankpage{
    \thispagestyle{empty}
    \addtocounter{page}{-1}
    \newpage}
\renewcommand\footrulewidth{0.4pt}

\begin{document}

\problemlist{Algorithms HW } 

%------------------------- Problem 1 -----------------------

\begin{problem}[1]
	\hfill
\end{problem}

\begin{solution}
\end{solution}

\newpage

%------------------------- Problem 2 -----------------------

\begin{problem}
	\includegraphics[scale=0.8]{2.png}
	\hfill
\end{problem}

\begin{solution}
If at some point $z_0 \in \C$ we have some power series coefficient
$c_n = 0$ then, recalling that the formula for the power series
coefficients, we have \[
	f^{(n)}(z_0) = 0,
\]
for some $n$.
Let us cover $\C$ with sets where the $n$th derivative is zero.
Let $Z_n := \{z_0 \in \C: f^{(n)} = 0\}$ be the set where the $n$th
derivative of $f$ is zero. Then, the statement in the question can be
rewritten as \[
	\C = \bigcup_{n = 0}^\infty Z_n.
\]
However, there are only countably many such sets $Z_n$ yet $\C$ is an
uncountable set. This means there is at least one $\tilde n$ where
$Z_{\tilde n}$ is uncountable.

We claim that $Z_{\tilde n}$ then has a limit point.
Consider tiling $\C$ with closed unit squares. One of these squares
must contain uncountably many points of $Z_{\tilde n}$, otherwise we
would have shown that $Z_{\tilde n}$ is countable.
That is, we have some square, a closed and bounded subset of $\C$,
which contains uncountably many points of $Z_{\tilde n}$. It follows 
by Bolzano-Weierstrass that $Z_{\tilde n}$ has an accumulation point
in that square. But in particular, $Z_{\tilde n}$ has an accumulation
point.

Now recall that $Z_{\tilde n}$ are the points where $f^{(\tilde n)}$
vanish. We have found a sequence of points in $\C$ converging to some 
limit point in $\C$ where $f^{(\tilde n)} = 0$. It follows then by the
identity theorem for holomorphic functions that $f^{(\tilde n)}$ is
identically zero.
Moreover, it then follows that $f^{(\tilde n + k)}$ is also
identically zero for all $k \in \Z_+$. In other words we have that the 
power series coefficients $c_n = 0$ for all $n \geq \tilde n$. 

Now we can write $f$ as a polynomial by repeatedly writing down the
antiderivatives for $f^{(\tilde n)}$. We can determine the polynomial
coefficients by inspecting the power series at $z = 0 $. 

\end{solution}

\newpage

%------------------------- Problem 3 -----------------------

\begin{problem}
	\includegraphics[scale=0.8]{3.png}
	\hfill
\end{problem}
\begin{solution}
\wg{Hmmm this might be bollocks, in particular, $\sum_i \abs{f_i(z)}$
may not be holomorphic}

In this situation the Cauchy integral formulae give us the following
\begin{align*}
	\frac{d}{dz}(\sum_i \abs{f_i(z)}) 
	&= \frac{1}{2\pi i } \int_\gamma \frac{\sum_i \abs{f_i(z)}}{(z - \xi)^2} d\xi \\
	&= \frac{M}{2\pi i } \int_\gamma \frac{1}{(z - \xi)^2} d\xi \\
\end{align*}
Suppose that $\gamma$ is a circle of radius $R$ centered 
at $z$. Then $1/(z-\xi)^2$ has an antiderivative on the punctured domain
bounded by $\gamma$. By the fundamental theorem of calculus the above
integral is then $0$.
Then we have. \[
	\frac{d}{dz}(\sum_i \abs{f_i(z)})  = 0. 
	\]
	By linearity of the derivative, this is only true when
	$\frac{d}{dz}\abs{f_i(z)} = 0$ for each $i$. And then this implies
	that each $f_i$ is constant on $U$. 
\end{solution}

\newpage

%------------------------- Problem 4 -----------------------

\begin{problem}
	\includegraphics[scale=0.8]{4.png}
	\hfill
\end{problem}

\begin{solution}
We want to construct $f(x+iy) = u(x,y) + iv(x,y)$ which is holomorphic
on $\Delta$. Recall that this means we need both $u,v$ to be
differentiable as multivariable real functions. Moreover, we need
$u,v$ to satisfy the Cauchy-Riemann equations \[
	\frac{\partial u}{\partial x} = \frac{\partial v}{\partial y} 
	\qquad \frac{\partial u}{\partial y} = - \frac{\partial v}{\partial x}.
\]
Recall from multivariable calculus that we can take single-variable
integrals to partially recover a function from its derivatve by taking
indefinite integrals
\begin{align*}
	\int \left( \pdv{v}{x} \right) dx &= v(x,y) + C(x) \\
	\int \left( \pdv{v}{y} \right) dy &= v(x,y) + D(y),
\end{align*}
for any differentiable functions $C,D : \R \to \R$. \wg{What are the bounds in this
set up?}

Here, we are given $u(x,y)$ and so we can paritally determine $v$ by
computing 
\begin{align*}
	v(x,y) + C(x) &= \int \left( -\pdv{u}{y} \right) dx \\
	v(x,y) + D(y) &= \int \left( \pdv{u}{x} \right) dy.
\end{align*}
However, this gives us a system of two equations with three unknowns. 
And so we can only ever determine $v$ up to either a function of $x$
or a function of $y$. 

Note that this process of integrating our partial derivatives to
reconstruct $v$ works (locally) exactly when 
$\pdv{}{x}(\partial v / \partial y)
= \pdv{}{y}(\partial v / \partial x)$ (this is tantamount to
``partials commuting''). If $v$ is a
twice differentiable real function then its partials must commute.
The harmonic condition on $u$ satisfies this condition in our case;
consider 
\begin{align*}
	\frac{\partial^2 v}{\partial y \partial x}
	&= \pdv{}{y} \left(-\pdv{u}{y} \right) && \text{Cauchy-Riemann} \\
	&= -\frac{\partial^2 u}{\partial y^2} \\
	&= \frac{\partial^2 u}{\partial x^2}  && \text{Harmonic condition on $u$} \\
	&= \pdv{}{x} \pdv{u}{x} \\
	&= \pdv{}{x} \pdv{v}{y} && \text{Cauchy-Riemann} \\
	&= \frac{\partial ^2 v}{\partial x \partial y}. \\
\end{align*}

\end{solution}

\newpage

\end{document}
