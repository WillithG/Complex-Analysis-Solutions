\documentclass[12pt,letterpaper,boxed]{hmcpset}
\usepackage[margin=1in,headheight=14pt]{geometry}
\usepackage{amsfonts, amsmath, amssymb, enumerate, fancyhdr, gensymb, lastpage, mathtools, parskip, graphicx}
\usepackage{xcolor, tikz-cd}
\newcommand{\wg}[1]{\textcolor{violet}{#1}}
\newcommand{\OO}{\mathcal O}
\newcommand{\Q}{\mathbb Q}
\newcommand{\R}{\mathbb R}
\newcommand{\C}{\mathcal C}
\newcommand{\Z}{\mathbb Z}
\newcommand{\abs}[1]{\left|#1\right|}
\newcommand{\im}{\text{im }}
\newcommand{\inv}{^{-1}}
\newcommand{\normal}{\unlhd} %% one can also use \trianglelelefteq
\newcommand{\anglee}[1]{\langle #1 \rangle}
\usepackage[shortlabels]{enumitem}

% Numbering macros
\pagestyle{fancy}
\lhead{Will Gilroy}
\chead{Complex Homework \#2}
\rhead{06 February 2026}
\lfoot{}
\cfoot{}
\rfoot{Page\ \thepage\ of\ \pageref{LastPage}}

\linespread{1.5}

\newcommand\blankpage{
    \thispagestyle{empty}
    \addtocounter{page}{-1}
    \newpage}
\renewcommand\footrulewidth{0.4pt}

\begin{document}

\problemlist{Complex Analysis Homework \#2} 

%------------------------- Problem 1 -----------------------

\begin{problem}
	\includegraphics[scale=0.8]{1.png}
	\hfill
\end{problem}

\begin{solution}
Following the definition, for $z,w \in \C$ we have the power series \[
	exp(z + w) = \sum_{n \geq 0} \frac{(z+w)^n}{n!},
\]
which expands to \[
	exp(z + w) = \sum_{n \geq 0} \frac{1}{n!}\sum_{m=0}^{n}
	\binom{n}{m} z^m w^{n-m}.
\]
Collect the terms containing $z^2$. We have \[
	\mathit{n=2:} \qquad
	\frac{1}{2!} \binom{2}{2} z^2 +
	\frac{1}{3!} \binom{3}{2} z^2 w +
	\frac{1}{4!} \binom{4}{2} + \cdots.
\]
Factoring out $z^2/2!$ gives
\begin{align*}
	n=2: \qquad 
	\frac{z^2}{2!} \displaystyle \sum_{m \geq 2} \binom{m}{2} \frac{2!}{m!} w^{m-2} 
	&= \frac{z^2}{2!} \displaystyle \sum_{m \geq 2} \frac{m!}{2!(m-2)!} \frac{2!}{m!} w^{m-2} && \text{factorial definition of binom}\\
	&= \frac{z^2}{2!} \sum_{m \geq 2} \frac{1}{(m-2)!} w^{m-2} \\
	&= \frac{z^2}{2!} \sum_{m \geq 0 } \frac{1}{m!} w^{m} && \text{Re-indexing sum}\\
	&= \frac{z^2}{2!} \cdot exp(w).
\end{align*}
Repeating this argument for the other $z$ terms of degree $n$ gives \[
	exp(z+w) = \sum_{n \geq 0} \frac{z^n}{n!} exp(w) = exp(z)exp(w).
\]

\end{solution}

\newpage

%------------------------- Problem 2 -----------------------

\begin{problem}
	\includegraphics[scale=0.75]{2.png}
	\hfill
\end{problem}

\begin{solution}
We can show that the given function is $C^\infty$ by induction on 
$n$ in $f^{(n)}(x)$. We claim that \[
	f^{(n)}(x) = \begin{cases}
		0 & x \leq 0 \\
		p(1/x)e^{-1/x^2} & x > 0
	\end{cases},
\]
where $p(1/x)$ is some polynomial in $1/x$. 
We show that $f \in C^1$. The only place we need to check
differentiability is at the origin. The limit of the difference
quotients approaching the origin from the left is $0$. 
From the right we have \[
	\lim_{h \to 0^+} \frac{f(h) - f(0)}{h}
	= \lim_{h \to 0^+} \frac{e^{-1/h^2}}{h}
	= 0,
\]

where the final limit can be seen by applying the change of variables
$t = 1/h^2$ and then using L'Hopital's rule.
It follows that the first derivative exists at $0$ and evaluates
to $0$. Then using the chain rule for all other $x$ we have that 
\[
	f'(x) = \begin{cases}
		0 & x \leq 0 \\
		\frac{2}{x^3} e^{-1/x^2} & x > 0
	\end{cases}.
\]

For the induction step we can use a similar limit argument to check that the
$n$th derivative exists and is equal to $0$ at $x = 0$. Moreover, the
product rule will give the claimed form of the $n$th derivative.

We have shown that $f \in C^\infty$. The discussion above also shows
that $f^{(n)}(0) = 0$ for all $n \geq 0$. And so, it follows that
the power series expansion about the origin is $f(x) \approx 0$. 
However, $f$ is not identically zero on any neighbourhood of $0$, in
other words
there does not exist a neighbourhood about $0$ where the power 
series expansion equals the function. Thus $f$ does not have a
convergent power series about $0$.

I think the takeaway from this question is to see an example of a smooth function $\R \to
\R$ which is not analytic, with the intuition being that it goes to
zero at the origin slower than any polynomial. 
I think Ben alluded to the idea that we'll see that holomorphic
functions always have a convergent power series over the domain where
it is holomorphic. And so I suppose this same thing cannot happen in
the complex case.

\end{solution}

\newpage

%------------------------- Problem 3 -----------------------

\begin{problem}
	\includegraphics[scale=0.8]{3.png}
	\hfill
\end{problem}
\begin{solution}
\begin{enumerate}[(a)]
\item Let $f(z) = \sum_{n \geq 0} nz^n$. Recalling the radius of 
convergence theorem, we have that $1/R = L = \limsup \abs{a_n}^{1/n}$.
Consider the following, we want to evaluate \[
	L = \lim_{m \to \infty}\sup_{n \geq m} n^{1/n} 
	= \lim_{m \to \infty} m^{1/m},
\]
Using a logarithm transformation and L'Hopital's rule we find that 
\begin{align*}
	\ln L &= \lim_{m \to \infty} \frac{\ln m}{m} \\
		&= \lim_{m \to \infty} \frac{1/m}{1} \\
		&= 0. 
\end{align*}
And so $L = e^{0} = 1 = 1/R$. Thus the power series $f$ converges absolutely
for all $\abs{z} < R = 1$. 

\wg{Let purple indicate some string of thoughts that I do not feel
completely resolved on.}
\wg{In principle, I think we also need to check for all the 
terms of $\abs z = 1$. For $z\in \R$ and $\abs z \geq 1$ this sum
definitely diverges}
\wg{
	Write $z = r \exp(i \theta)$ with $r = 1$. 
	Then our power series becomes 
	$f(z) = \sum_{n \geq 0} n \exp(in\theta)$, 
	the exponential term only adding a phase to each term.
	I think this series then diverges for all $\abs z = 1$. 
}

Now suppose $f(z) = \sum_{n \geq 0} z^n /n^2$. Using similar reasoning
we want to evaluate \[
	L = \lim_{m \to \infty} \sup_{n \geq m} n^{-2/n} 
		= \lim_{m \to \infty} m^{-2/m}.
\]
Using the same $\ln$ computation as above, we find that
\begin{align*}
	\ln L &= \lim_{m \to \infty}\left( \frac{-2}{m}\ln m \right) = 0 \\
	L &= 1 \implies R = 1. 
\end{align*}
Hence the power series $f$ converges absolutely for all $\abs {z} <
1$.
\wg{again, should we check what happens at $R = 1$. Perhaps it depends 
on choice of $z$?}

\item 
Let $f(z) = \sum_{n \geq 1} z^n / n$. 
Notice that $f(1) = \sum_{n \geq 1} 1/n$, this is the harmonic series
and so diverges.
On the other hand $f(-1) = \sum_{n \geq 1} (-1)^n /n$ is the
alternating harmonic series and converges to $-\ln(2)$ (the minus sign
being to do with where I stared indexing the sum above).
\wg{The value of this series can be found by considering the Taylor
expansion of $\ln(1 + x)$}. 

\end{enumerate}
\end{solution}

\newpage

%------------------------- Problem 4 -----------------------

\begin{problem}
	\includegraphics[scale=0.8]{4.png}
	\hfill
\end{problem}

\begin{solution}
Recall that the power series coefficients of an infinitely
differentiable function, at a point $z_0 \in \C$, is given by \[
	a_n = \frac{f^{(n)}(z_0)}{n!}.
\]
There exist neighbourhoods of $z = 1$ which do not include the origin
and which do not include the branch cut along the negative part of the
real axis.
And so $\log z$ is differentiable on some open set about $z = 1$ with
derivative $1/n$. It follows then that the coefficients of the power
series for $\log z$ about $z = 1$ for $n \geq 1$ are given by \[
	a_n = (-1)^{n-1} \frac{(n-1)!}{n!} (\frac{1}{z^n})\vert_{z = 1}
		= (-1)^{n-1} \frac{1}{n}.
\]
And then, recalling that $\log 1 = 0$, our power series expansion about $z = 1$ is \[
	\log z = \sum_{n \geq 1} \frac{(-1)^{n-1}}{n} (z-1)^n
\]
\end{solution}

\newpage

%------------------------- Problem 5 -----------------------

\begin{problem}
	\includegraphics[scale=0.8]{5.png}
	\hfill
\end{problem}

\begin{solution}
\wg{
	Apologies, I didn't make a huge amount of headway into this
	problem. Two questions prior we found a few power series whose
	radii of convergence was $R = 1$ (i.e. power series which 
	converge absolutely on the unit disc).
}
\wg{
	However, I was playing a bit instead with the geometric series
	$f(z) = \sum_{n \geq 0} z^n$. However, I had some trouble 
	understanding how to formualte the constraints of having
	a function which is equal to the geometric
	series on the unit disc, and which is holomorphic. I'll probs 
	come chat about this one at office hours.
}

\end{solution}

\newpage

%------------------------- Problem 6 -----------------------

\begin{problem}
	\includegraphics[scale=0.8]{6.png}
	\hfill
\end{problem}

\begin{solution}
Recall that a complex function $f: \C \to \C$ 
is conformal at $z \in \C$ if and only if it is holomorphic at $z$ and
$f'(z) \neq 0$.
For $f = \exp$ we have that $f$ is entire with $f'(z) = \exp(z)$. 
Moreover, $f'(a+bi) = \exp(a + bi) = \exp(a)\exp(ib)$. In particular,
$\abs{f'(z)} = \abs{\exp(a)} > 0$ for $a \in \R$, the inequality being
strict.
That is, $f'(z) \neq 0$ for all $z \in \C$ and so $f$ is conformal for
all $z \in \C$. 


\end{solution}

\newpage



\end{document}
