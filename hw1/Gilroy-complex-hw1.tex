\documentclass[12pt,letterpaper,boxed]{hmcpset}
\usepackage[margin=1in,headheight=14pt]{geometry}
\usepackage{amsfonts, amsmath, amssymb, enumerate, fancyhdr, gensymb, lastpage, mathtools, parskip, graphicx}
\usepackage{xcolor, tikz-cd}
\newcommand{\wg}[1]{\textcolor{violet}{#1}}
\newcommand{\OO}{\mathcal O}
\newcommand{\Q}{\mathbb Q}
\newcommand{\R}{\mathbb R}
\newcommand{\C}{\mathbb C}
\newcommand{\Z}{\mathbb Z}
\newcommand{\im}{\text{im }}
\newcommand{\inv}{^{-1}}
\newcommand{\normal}{\unlhd} %% one can also use \trianglelelefteq
\newcommand{\anglee}[1]{\langle #1 \rangle}
\newcommand{\pdv}[2]{\frac{\partial #1}{\partial #2}}
\newcommand{\abs}[1]{\left\vert #1 \right\vert}
\usepackage[shortlabels]{enumitem}

% Numbering macros
\pagestyle{fancy}
\lhead{Will Gilroy}
\chead{Complex Homework \#1}
\rhead{28 January 2026}
\lfoot{}
\cfoot{}
\rfoot{Page\ \thepage\ of\ \pageref{LastPage}}

\linespread{1.5}

\newcommand\blankpage{
    \thispagestyle{empty}
    \addtocounter{page}{-1}
    \newpage}
\renewcommand\footrulewidth{0.4pt}

\begin{document}

\problemlist{Complex Analysis Homework \#1 } 

%------------------------- Problem 1 -----------------------

\begin{problem}
	\includegraphics[scale=0.8]{1.png}
	\hfill
\end{problem}

\begin{solution}
Let us consider some special cases.
Let us first take one of our fixed points to be $0$,
say $v = 0$. Let us also study the case where $\rho = 1$ to begin.
Our defining equation becomes $\abs{z} = \abs{z - w}$,
notice that $z$ satisfies this system if and only if it satsifies
$\abs{z}^2 = \abs{z-w}^2$.
Let us write $z = a + bi$ and $w = c + di$ then \[
	\abs{z} = \abs{z - w}
	\iff c^2 + d^2 - 2ac - 2bd = 0.
\]
Now let us consider some cases for $w$. If $w$ is real and non-zero
then $w = c$ and we can solve for $a = \frac{1}{2} c$. 
$b$ is free and so the locus of points which solve this case is 
$z = \frac{1}{2}c + bi$ for any fixed $c \in \R$ and for all $b \in
\R$. This is exactly a vertical line of points. In particular,
it is the vertical line of points through the origin, translated by $\frac{1}{2}c$.

Performing an extremely similar calculation, if we consider the case where $w$ is purely
imaginary, $w = di$ for some $d \in \R$, then our locus of points will
be $z = a + \frac{1}{2}d$ for some for all $a \in \R$. These are
horizontal lines. In particular, these are the horizontal line of
points through the origin, translated by $\frac{1}{2}d$

In the case where $w = c + di$ with $c,d$ non-zero. We are now looking
for points which have the same distance from the origin after being
translated by $-(c + di)$. If $c = d = 1$ then our locus of points is
a diagonal line. In particular, it is the diagonal line passing
through the origin and $-(1+i)$, translated\footnote{
	Another way of saying this is that it's the line defined by the 
	normal vector $[1, 1]$ and then translated by the vector 
	$\frac{1}{2}[1,1]$. 
} by $\frac{1}{2}(1+i)$. 
Geometrically, the $\rho = 1$ case is the locus of lines which are
equidistant from $0$ and $1 + i$. 
Likewise, following a similar argument, for general $c,d$ we have that the locus of points is some 
diagonal line.

The entire discussion above holds for $v \neq 0$ but still $\rho = 1$,
Except now the defining horizontal/vertical/diagonal lines are those
relative to $v$, rather than relative to the origin.

Now we consider the $\rho \neq 1$ case.
It turns out we can show that $\rho \neq 1$ gives a circle by
analyzing the conditions directly.
Suppose that we have $\rho \neq 1$ and $v, w \in \C$ 
arbitrary. There exists an isometry which translates $v$ to the
origin, and which rotates $w$ to be on the real axis. 
And so without loss of generality we can take $v = 0$ and $w \in \R$. 
I will analyze the case where $w = 1$ to simplify the arithmetic, but the
argument for other real $w$ is similar to the following.
With these transformations our condtion becomes 
\[
	\abs{z}^2 = \rho \abs{z - 1}^2
\]
If we let $z = a + bi$ and expand the definitions, we have
\begin{align*}
	a^2(1 - \rho) + b^2(1- \rho) &= -2\rho a + \rho \\
	\left(a^2 + \frac{2\rho}{1-\rho} a\right) + b^2 &= \frac{\rho}{1- \rho} && \rho \neq 1 \\
	\left( a + \frac{\rho}{1 - \rho} \right)^2 + b^2 &= \frac{\rho}{1 - \rho} + \frac{\rho^2}{(1-\rho)^2} && \text{Completing the square} \\
	\left( a + \frac{\rho}{1 - \rho} \right)^2 + b^2 &= \frac{\rho}{(1-\rho)^2}.
\end{align*}
This is manifestly a circle of radius
$(\sqrt\rho)/(1-\rho)$. \wg{Note, I think I may have missed an extra
factor of $\rho$ in my starting condition.}

\end{solution}

\newpage

%------------------------- Problem 2 -----------------------

\begin{problem}
	\includegraphics[scale=0.7]{2.png}
	\hfill
\end{problem}

\begin{solution}
We show that there is no total ordering on $\C$ by considering the
elements $1, i, 0 \in \C$.
Suppose there is a total order on $\C$ denoted $\prec$
Since $1,i,0$ are distinct, we have four possible cases to consider:
$(1 \prec 0, i \prec 0), (1 \prec 0, i \succ 0), (1 \succ 0, i \prec 0),
\text{ or } (1 \succ 0, i \succ 0)$. We show that we arrive at a
contradiction in all cases.

$(1 \succ 0, i \succ 0):$ Since $i \succ 0$ we have the following
\begin{align*}
	1 \succ 0 &\implies i \succ 0 && \text{Property $(c)$} \\
		&\implies -1 \succ 0 && \text{Property $(c)$} \\
		&\implies 0 \succ 1 && \text{Property $(b)$},
\end{align*}
a contradiction.

$(1\succ 0, i \prec 0):$ Since $i \prec 0$ we have that $0 \prec -i$
by property $(b)$. Then, by property $(c)$ and then $(b)$, we have the following
\begin{align*}
	-i \succ 0 \implies -1 \succ 0 \implies 0 \succ 1, 
\end{align*}
a contradiction.

$(1 \prec 0, i \succ 0):$ Using similar reasoning, with $i \succ 0$, applying property $(c)$ a
sufficient number of times with $w = i$ implies $1 \succ 0$. A contradiction.

$(1 \prec 0, i \prec 0):$. Using similar reasoning, $i \prec 0$
implies $-i \succ 0$. Then applying property $(c)$ a sufficient number
of times with $w = -i$ gives $1 \succ 0$. A contradiction. 
\end{solution}

\newpage

%------------------------- Problem 3 -----------------------

\begin{problem}
	\includegraphics[scale=1]{3.png}
	\hfill
\end{problem}
\begin{solution}
Suppose we have a function $f: \C \to \C$ defined by
$f(z = x+iy) = u(x,y) + iv(x,y)$ where $x,y \in \R$.
Then recall the Cauchy-Riemann equations are given by \[
	\pdv{u}{x} = \pdv{v}{y} \qquad \pdv{v}{x} = -\pdv{u}{y}.
\]

Recalling the polar form of a complex number $z = re^{i \theta} =
r\cos(\theta) + i\sin(\theta)$ for $r > 0$. We can read the variable
transformation
$x = r\cos(\theta)$ and $y = r\sin(\theta)$. 
Being pedantic about our notation (we will abuse at the end), let us write
\[
	\tilde u(r, \theta) := u(r \cos(\theta), r\sin(\theta)) 
	\qquad 
	\tilde v(r, \theta) := v(r\cos(\theta), r\sin(\theta)).
\]
Now, recalling the partial derivative chain rule for multivariable
functions we have 
\begin{align*}
	\pdv{\tilde u}{r} &= \pdv{u}{x} \cos(\theta) + \pdv{u}{y}\sin(\theta) \\
	\pdv{\tilde u}{\theta} &= \pdv{u}{x}(-r\sin(\theta)) + \pdv{u}{y}r\cos(\theta) \\
	\pdv{\tilde v}{r} &= \pdv{v}{x}\cos(\theta) + \pdv{v}{y}r\cos(\theta) \\
	\pdv{\tilde v}{\theta} &= \pdv{v}{x}(-r\sin(\theta)) + \pdv{v}{y}r\cos(\theta).
\end{align*}
Now, using our original Cauchy-Riemann equations, we can derive
relations between these equations.
Consider, for example 
\begin{align*}
	\frac{1}{r} \pdv{\tilde v}{\theta} &= \pdv{v}{x}(- \sin\theta) + \pdv{v}{y}(\cos \theta) 
			&& \text{Using the formulae above,}\\
		&= \pdv{u}{y}(\sin \theta) + \pdv{u}{x}(\cos \theta) && \text{Using Cauchy-Riemann,}\\
		&= \pdv{\tilde u}{\tilde r}.
\end{align*}
Using very similar reasoning will also give us that $(1/r)(\partial
\tilde u/\partial \theta) = -(\partial \tilde v/ \partial r)$. 
And now, we abuse notation by setting $u(r,\theta) = \tilde u(r,
\theta)$ and $v(r, \theta) = \tilde v(r, \theta)$. 

\item 
Recall that our function $f(z) = u(x,y) + iv(x,y)$ is holomorphic if
and only if $f$ is real differentiable (implying all the relevent
partial derivatives exist) and $u,v$ satisfy the Cauchy-Riemann
equations. We verify that $\log z : \C \to \C$ is holomorphic where
it's defined, by checking if it satisfies the Cauchy-Riemann
equations.

For $\log z$ we have $u(r,\theta) = \log r$ and $v(r, \theta) =
\theta$. Then notice we have 
\begin{align*}
	\pdv{u}{r} &= \frac{1}{r} = \frac{1}{r} \cdot 1 = \frac{1}{r} \pdv{v}{r} \\
	\frac{1}{r} \pdv{u}{\theta} &= 0 = - \pdv{v}{r},
\end{align*}
And so, indeed, the complex logarithm function is holomorphic where it
is defined.
\end{solution}
\newpage

%------------------------- Problem 4 -----------------------

\begin{problem}
	\includegraphics[scale=0.8]{4.png}
	\hfill
\end{problem}

\begin{solution}
Writing $f(x+iy) = u(x,y) + iv(x,y)$ we can read that $u(x,y) =
\sqrt{\abs{x}\abs{y}}$ and $v(x,y) = 0$.

First we need to verify that the partial derivatives for $u,v$ exist
at $(0,0)$. Since $v = 0$ both its partial derivatives exist and evaluate
to zero everywhere. Now consider $u$. Recalling the definition for
partial derivative, consider 
\[
	u_x(0,0) := \lim_{h \to 0} \frac{u(h, 0) - u(0,0)}{h}
	= \lim_{h \to 0} 
		\frac{\sqrt{\abs{h}\abs{0}} - \sqrt{0}}
		{h} 
	= 0,
\]
And so the partial derivative $u_x$ exists at $(0,0)$ and evaluates to
$0$. A similar calculation will show that the partial derivative
$u_y$ exists and evaluates to $0$ at $0$.

Now we can check if $u,v$ satisfy the Cauchy-Riemann equations 
at $(0,0)$.
Recall that the Cauchy-Riemann equationsa are given by \[
	\pdv{u}{x} = \pdv{v}{y} \qquad \pdv{v}{x} = -\pdv{u}{y}.
\]
Since all relevent partial derivatives evaluate to $0$ at $z = 0$, 
the Cauchy-Riemann equations are sastisfied at $z = 0$.
The fact that the Cauchy-Riemann equations are satisfied shows that
the difference quotients of our function along the real and imaginary
axes about $(0,0)$ agree.

However, we show that this function is not holomorphic by constructing
two sequences of difference quotients approaching $z = 0$ which disagree.

Consider a sequence along the real line which approaches $z = 0$ (such as
$\{1/n\}_{n=1}^{\infty}$). Given this sequence of complex numbers we
have the difference quotients are all $0$:
\[
	\frac{f(z_n) - f(0)}{z} 
	= \frac{ \sqrt{\abs{1/n}\abs{0}} - 0}{1/n} = 0.
\]
And so the limit of the difference quotients, with respect to this
sequence, exists and evaluates to $0$.

Now, consider a sequence of complex numbers approaching $0$ via the
ray which passes through $0$ and $1 + i$. One such sequence is 
$\{1/n + i (1/n)\}$. Our difference quotients are now\[
	\frac{f(z_n) - f(0)}{z_n} 
	= \frac{ \sqrt{\abs{1/n} \abs{1/n}} }{1 + i (1/n)} 
	= \frac{1}{1+i} 
	= \frac{1}{2}(1 - i),
\]
	for all $n$. 
	It follows that the limit of the difference quotients with respect
	to this sequence exists and evalutes to $0.5(1 - i)$.	

We have then shown that the limit
	\[
	\lim_{z \to 0} \frac{f(z) - f(0)}{z}
\]
	does not exist, since it is not well-defined with respect to
	choice of limit approach.
	And so, $f$ is not differentiable at $0 \in \C$ by
	definition.

I think this argument shows that, whilst all partial derivatives for
$u,v$ exist at $(x,y) = (0,0)$ and the Cauchy-Riemann equations are
satisifed, It must be that the given function is not
real-differentiable at $z = 0$. 

\end{solution}

\newpage

%------------------------- Problem 5 -----------------------

\begin{problem}
	\includegraphics[scale=0.8]{5.png}
	\hfill
\end{problem}

\begin{solution}
Let $f: U \to \C$ be given by $f(x+iy) = u(x,y) + iv(x,y)$ with
$u(x,y) = a$ for some $a \in \R$. Then both partial derivatives of $u$ vanish everywhere
on $U$, i.e. $\partial
u/ \partial x = 0$ and $\partial u / \partial y = 0$.
Since $f$ is holomorphic on $U$ we have that $f$ satisfies the
Cauchy-Riemann equations on $U$. And so we have \[
	\pdv{v}{x} = 0 \qquad \pdv{v}{y} = 0,
\]
everywhere on $U$, in other words $\nabla v = 0$.
Since $f$ is holomorphic on $U$ we have that $v(x,y)$ is
differentiable as a real function everywhere on $U$. 
It then follows that $v(x,y) = b$ for some $b \in \R$ as a real
function on $U$ (a result from the real-analysis of multivariate
functions\footnote{
	I thiiiiiink we might also need that $U$ is connected, but my
	multivariable real analysis is a bit rusty.
}).
Then, by definition, $f = u + iv$ is a constant function.
\end{solution}

\newpage

\end{document}
