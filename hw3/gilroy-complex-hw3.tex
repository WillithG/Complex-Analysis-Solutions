\documentclass[12pt,letterpaper,boxed]{hmcpset}
\usepackage[margin=1in,headheight=14pt]{geometry}
\usepackage{amsfonts, amsmath, amssymb, enumerate, fancyhdr, gensymb, lastpage, mathtools, parskip, graphicx}
\usepackage{xcolor, tikz-cd}
\newcommand{\wg}[1]{\textcolor{violet}{#1}}
\newenvironment{wgenv}{\color{violet}}{}
\newcommand{\OO}{\mathcal O}
\newcommand{\Q}{\mathbb Q}
\newcommand{\R}{\mathbb R}
\newcommand{\C}{\mathcal C}
\newcommand{\Z}{\mathbb Z}
\newcommand{\abs}[1]{\left|#1\right|}
\newcommand{\im}{\text{im }}
\newcommand{\inv}{^{-1}}
\newcommand{\normal}{\unlhd} %% one can also use \trianglelelefteq
\newcommand{\anglee}[1]{\langle #1 \rangle}
\usepackage[shortlabels]{enumitem}

% Numbering macros
\pagestyle{fancy}
\lhead{Will Gilroy}
\chead{Algs Homework \#}
\rhead{03 November 2021}
\lfoot{}
\cfoot{}
\rfoot{Page\ \thepage\ of\ \pageref{LastPage}}

\linespread{1.5}

\newcommand\blankpage{
    \thispagestyle{empty}
    \addtocounter{page}{-1}
    \newpage}
\renewcommand\footrulewidth{0.4pt}

\begin{document}

\problemlist{Algorithms HW } 

%------------------------- Problem 1 -----------------------

\begin{problem}
	\includegraphics[scale=0.8]{1.png}
	\hfill
\end{problem}

\begin{solution}
We start by splitting the integrand with partial fractions.
Notice that \[
\frac{1}{(x-a)(x-b)} = \frac{1}{(a-b)(x-a)} - \frac{1}{(a-b)(x-b)}.
\]
Next we paramaterize $S_r = \gamma(t) = r\exp(it)$ with $\abs a < r <
\abs b$.
Now, computing our integral directly gives
\begin{align*}
	\int_{S_r}\frac{1}{(z-a)(z-b)} dz 
	&= \int_{t=0}^{2\pi} f(\gamma(t)) \gamma'(t) dt \\
	&= \int \frac{i}{a-b} 
		\left( \frac{r \exp(it)}{r\exp(it) - a} - \frac{r \exp(it)}{r\exp(it) - b}\right) dt \\
	&= \frac{i}{a-b} \int_{t = 0}^{2\pi} \left( \frac{1}{1 - \frac{a}{r}\exp(-it)} - \frac{1}{1 - \frac{b}{r}\exp(-it)} \right) dt
\end{align*}
\wg{if we can show that the integrand is $1$ then we are done}
\end{solution}

\newpage

%------------------------- Problem 2 -----------------------

\begin{problem}
	\includegraphics[scale=0.8]{2.png}
	\hfill
\end{problem}

\begin{solution}
Notice that $Re(e^{ix^2}) = \cos(x^2)$ for $x \in \R$. 
To that end we consider integrating $\int_\gamma f(z) dz$
where $f(z) = e^{-z^2}$ over 
the following contour.
\wg{insert contour}

\begin{wgenv}
Notice that $f'(z)$ is entire with $f'(z) = 2ize^{iz^2}$, and so 
also has continuous derivative. By Cauchy's Theorem we have that 
$\int_\gamma f(z) dz = 0$ for all $R$. The integral we are interested 
in is $\lim_{R \to \infty} Re \int_{A} f(z) dz$. To that end,
we compute the integral over the other parts of the curve.

Consider the integral along $C$. We can parameterize this part of 
the curve as $\gamma(t) = t e^{i \pi/4}$ with $t: R \to 0$. 
Then consider 
\begin{align*}
	\int_C f(z) dz &= - \int_0^R f(\gamma(t)) \gamma'(t) dt \\
		&= - \int_0^R e^{i t^2(e^{i \pi/4})^2} e^{i \pi/4} dt \\
		&= - e^{i \pi / 4} \int_0^R e^{-t^2} dt
	\intertext{Then in the limit $R \to \infty$, }
	\int_C f(z) dz &= - e^{i \pi / 4} \frac{\sqrt \pi }{2}. 
\end{align*}
\end{wgenv}
We integrate along part $A$ of the curve above. We parameterize this
part of the curve at $\gamma(t) = t \in [0,R]$. Then we have 
\begin{align*}
	\lim_{R \to \infty} \int_A f(z) dt
	&= \lim_{R \to \infty} \int_{t=0}^R f(\gamma(t))\gamma'(t) dt \\
	&= \lim_{R \to \infty} \int_{t=0}^R e^{-t^2} dt \\
	&= \int_{t=0}^\infty e^{-t^2} dt \\
	&= \frac{\sqrt \pi}{2},
\end{align*}
as given in the book.

\wg{How do we show that the integral over part $B$ is zero?}

Notice that $f$ is entire with continuous derivative. By Cauchy's
theorem we have \[
	-\int_{C} f(z) dz = \int_A f(z) dz = \sqrt\pi / 2. 
\]
Consider the integral along $C$. We paramaterize this part of the
curve as $\gamma(t) = t e^{i \pi /4}$ with $t: R \to 0$.
Now consider
\begin{align*}
	-\int_C f(z) dz 
	&= -\int_{t=R}^0 f(\gamma(t)) \gamma'(t) dt \\ 
	&= e^{i\pi/4}\int_{t=0}^R e^{(t e^{i\pi/4})^2} dt \\
	&= e^{i\pi/4}\int_{t=0}^R e^{(\frac{t}{\sqrt 2} (1 + i))^2} dt \\
	&= e^{i\pi/4}\int_{t=0}^R e^{-it^2} dt \\
	&= e^{i\pi/4}\int_{t=0}^R \cos(-t^2) + i \sin(-t^2) dt \\
	&= e^{i\pi/4}\int_{t=0}^R \cos(t^2) - i \sin(-t^2) dt \\
	&= \frac{1}{\sqrt 2}(1 + i )\int_{t=0}^R \cos(t^2) - i \sin(-t^2) dt \\
	&= \frac{1}{\sqrt 2} \left[
		\int_0^R \cos(t^2) + \sin(t^2) dt 
		+ i\left(\int_0^R \cos(t^2) - \sin(t^2) dt \right)
	\right]	
\end{align*}
Now, taking the limit $R \to \infty$ and recalling Cauchy's formula
gives 
\[
	\int_0^\infty \cos(t^2) + \sin(t^2) dt 
	+ i\left(\int_0^\infty \cos(t^2) - \sin(t^2) dt \right)
	= \frac{\sqrt{2 \pi}}{2}
\]
Comparing the real and imaginary part of this equation gives
$\int_0^\infty \cos(t^2) dt = \int_0^\infty \sin(t^2) dt$ 
and then gives $\int_0^\infty \cos(t^2) dt = (\sqrt{2\pi})/4$




\end{solution}

\newpage

%------------------------- Problem 3 -----------------------

\begin{problem}[3]
	\hfill
\end{problem}
\begin{solution}
\end{solution}

\newpage

%------------------------- Problem 4 -----------------------

\begin{problem}[4]
	\hfill
\end{problem}

\begin{solution}
\end{solution}

\newpage

%------------------------- Problem 4 -----------------------

\begin{problem}[4]
	\hfill
\end{problem}

\begin{solution}
\end{solution}

\newpage

\end{document}
